%\title{Template for an unofficial Europass curriculum vitae}
% !TEX encoding = UTF-8
% !TEX program = pdflatex
% !TEX spellcheck = en_EN

% Europass curriculum vitae class
% Author: Andrea Zennaro
%
% based on the europecv class by:
% Nicola Vitacolonna (http://ctan.org/pkg/europecv)
% Giacomo Mazzamuto (http://www.ctan.org/pkg/europasscv)

% This material is subject to the LaTeX Project Public License Version 1.3.
% See  http://www.latex-project.org/lppl.txt
% for the details of that license

\newif\ifenglish
\newif\ifextraInfo

% \englishtrue
\englishfalse
\extraInfotrue
% \extraInfofalse

\ifenglish
  \documentclass[english,a4paper]{europasscv}
\else
  \documentclass[italian,a4paper]{europasscv}
\fi

\usepackage{comment}
\usepackage{eurosym}
\renewcommand\ThisComment[1]{%
        \immediate\write\CommentStream{\unexpanded{#1}}%
} 

%----------------------------------------------------------------------------------------
% SELECTOR FOR COMMENT PACKAGE
%----------------------------------------------------------------------------------------
% excludecomment means OFF
% includecomment means ON
\ifenglish
    \includecomment{en} % Italian language
    \excludecomment{it} % English language
\else
    \includecomment{it} % English language
    \excludecomment{en} % Italian language
\fi

\ifextraInfo
\includecomment{extraInfo} % To enable extra Additional Information and Annexes sections
\else
\excludecomment{extraInfo} % To enable extra Additional Information and Annexes sections
\fi

%----------------------------------------------------------------------------------------
%	NAME AND CONTACT INFORMATION SECTION
%----------------------------------------------------------------------------------------
\ecvname{Andrea Zennaro}
\begin{en}
\usepackage[english]{babel}
\ecvaddress{Via Concordia,14 31100 -- Treviso (Italia)}
\ecvtelephone{+39 0422 1740026}
\ecvworkphone{+39 349 3400835} 
\ecvmobile{+39 347 9812393}
\ecvlinkedinpage{it.linkedin.com/in/andrea-zennaro-4654915}
\ecvgender{Male}
\ecvnationality{Italian}
\ecvdateofbirth{30 January 1972}
\end{en}
\begin{it}
\usepackage[italian]{babel}
\ecvaddress{Via Concordia,14 31100 -- Treviso (Italia)}
\ecvtelephone{+39 0422 1740026}
\ecvworkphone{+39 349 3400835} 
\ecvmobile{+39 347 9812393}
\ecvlinkedinpage{it.linkedin.com/in/andrea-zennaro-4654915}
\ecvgender{Maschile}
\ecvnationality{Italiana}
\ecvdateofbirth{30 gennaio 1972}
\end{it}
\ecvemail{andrea.zennaro@gmail.com}

\ecvpicture[height=3.8cm]{Zennaro_Andrea.jpg}

\begin{document}
  \begin{europasscv}

  \ecvpersonalinfo

%----------------------------------------------------------------------------------------
% JOB APPLIED FOR SECTION
%----------------------------------------------------------------------------------------
\begin{en}
  \ecvbigitem{Job applied for}{IT Project Manager}
  % Other headings available:
  % Job applied for
  % Position
  % Preferred job
  % Studies applied for
  % Personal statement
\end{en}
    
\begin{it}
  \ecvbigitem{Occupazione desiderata}{IT Project Manager}
  % Other headings available:
  % Occupazione per la quale si concorre 
  % Posizione ricoperta
  % Occupazione desiderata
  % Titolo di studio
  % Dichiarazioni personali
\end{it}

%----------------------------------------------------------------------------------------
% WORK EXPERIENCE SECTION
%----------------------------------------------------------------------------------------
\begin{en}
  \ecvsection{Work Experience}
  \ecvtitle{May 2010 --  }{Client Manager - Project Manager - Account Manager }
  \ecvitem{}{Gruppo Euris S.p.a.\newline Via Uruguay, 54, 35127 Padova}
  \ecvitem{}{
    To be completed \ldots
    \begin{ecvitemize}
    \item One
    \item Two 
    \item Three
    \end{ecvitemize}
   }
  \ecvtitle{July 2009 -- May 2010}{Responsabile pianificazione delle risorse}
  \ecvitem{}{Gruppo Euris S.p.a.\newline Via Uruguay, 54, 35127 Padova}
  \ecvitem{}{
  Da completare \ldots
  }

  \ecvtitle{July 2007 -- July 2009}{Responsabile Tecnico per le filiali di Padova e Verona}
  \ecvitem{}{Gruppo Euris S.p.a.\newline Via Uruguay, 54, 35127 Padova}
  \ecvitem{}{
  Responsabile diretto di 45 dipendenti.\newline
  Gestione del colloquio tecnico ed inserimento delle nuove risorse nei gruppi di lavoro.\newline
  Analisi delle competenze tecniche per la fornitura dei consulenti da inserire nei progetti dei clienti.\newline
  Erogazione di corsi interni per la formazione ed il \textquotedblleft training on the job\textquotedblright{} delle risorse.\newline
  Analisi dei requisiti, analisi tecnica, stima dei tempi e dei costi, definizione di architetture software, gestione stato avanzamento lavori e change request, redazione dei vari documenti tecnici di progetto.\newline
  \ecvhighlight{Principali progetti gestiti}
  \begin{ecvitemize}
  \item Ott 2007, Redazione del documento tecnico per la partecipazione alla gara per la fornitura di un sistema software per la gestione del \textquotedblleft Sistema di Legalità\textquotedblright{} per la società InfoCamere di Padova.
  \item Feb - Mag 2008, Progettazione e Team Leader nella realizzazione di un modulo software per la gestione del \textquotedblleft Controllo di Qualità\textquotedblright{} nei processi produttivi del settore siderurgico sviluppato per un'azienda di automazioni industriali del Gruppo Marcegaglia.
  \item Mar -- Ago 2008, Progettazione e coordinamento di 4 sviluppatori nelle attività di sviluppo di un modulo di \textquotedblleft System Integration\textquotedblright{} per il sistema CRM implementato presso la società CRIF di Bologna, operante del settore Finance.
  \item Set -- Ott 2008, Progettazione e sviluppo di una Web Application per la compilazione della modulistica telematica per la gestione di rifiuti pericolosi e l'emissione di sostanze inquinanti per conto di un'azienda operante nel settore \textquotedblleft Ecologia e Ambiente\textquotedblright{} (area Pubblica Amministrazione).
  \item Nov 2008 -- Mar 2009, Consolidamento ed estensione delle funzionalità del modulo di integrazione del CRM di CRIF permettendo la gestione di informazioni provenienti da funzioni aziendali non previste nella prima fase del progetto ed estensione delle piattaforme applicative  coinvolte nel processo.
  \item Apr -- Lug 2009, Progettazione e coordinamento delle attività di sviluppo del sistema di gestione degli \textquotedblleft Incidents\textquotedblright{} nel progetto di CRM per CRIF. Realizzazione di un modulo per la gestione ed il monitoraggio del \textquotedblleft Service Level Agreement\textquotedblright.
  \end{ecvitemize}
  }
  \ecvtitle{January 2005 -- July 2007}{Senior Consultant}
  \ecvitem{}{Engineering S.p.a.\newline Via G. del Pian dei Carpini, 1, 50100 Firenze}
  \ecvitem{}{
  Consolidamento delle procedure di alimentazione dell'Enterprise Data Warehouse in ambiente
  dipartimentale presso il Consorzio Operativo del Gruppo Monte dei Paschi di Siena (COGMPS).\newline
  Coordinamento del gruppo di lavoro impegnato nello sviluppo degli strumenti di analisi e fruizione dei dati del EDW presso la sede del COGMPS:
  \begin{ecvitemize}
  \item Giu 2006 -- Giu 2007, Analisi, progettazione e sviluppo del Sistema di Indagine OLAP integrato. L'applicazione è stata sviluppata con metodologia agile e sviluppo iterativo utilizzando la modellazione UML per la descrizione degli \textsl{Use Cases}, degli \textsl{Activity Diagram}, dei \textsl{Sequence Diagram} e dei \textsl{Class Diagram}. Durante la progettazione è stato fatto uso dei principali design pattern, in particolare è stato applicato il CAB (Composite Application Block) definito da Pattern\&Practice di Microsoft. L'applicazione è stata realizzata in ambiente .Net, con interfaccia grafica Window Forms e business layer realizzato con tecnologia COM+.
  \item Set 2005 -- Giu 2006, Analisi e progettazione del gestore del modello dati della Banca. Applicazione di tipo Smart Client, che consente di descrivere la mappatura degli attributi delle entità che rappresentano il modello dati concettuale della banca verso le diverse rappresentazioni del modello dati relazionale implementato nelle applicazioni di ogni area.
  \item Gen 2005 -- Set 2005, Progettazione e sviluppo di un modulo software per l'esecuzione di indagini OLAP che richiedono l'uso di dati calcolati dal motore di calcolo del sistema di Batch Processing. Il sistema realizzato con tecnologia ASP.NET è in grado di inviare una richiesta ad modulo COM+ che permette di istanziare un processo batch \textquotedblleft{}on demand\textquotedblright{} e restituire i dati calcolati all'applicazione web.
  \end{ecvitemize}
  }
  \ecvtitle{July 2003 -- December 2004}{Technical Leader for Business Intelligence Application}
  \ecvitem{}{Trend S.p.A.\newline Via Sorbanella, 30, 25125 Brescia}
  \ecvitem{}{
  Progettazione e coordinamento delle attività di sviluppo di applicazioni per la realizzazione di data warehouse.\newline
  Progettazione di applicazioni di Business Intelligence per il settore bancario e finanziario.
  }
  \ecvtitle{March 2001 -- June 2003}{Senior Software Engineer}
  \ecvitem{}{EURIS S.r.l.\newline Via Uruguay, 53, 35100 Padova}
  \ecvitem{}{
  Attività di analisi, progettazione, sviluppo e consolidamento della piattaforma di sviluppo Itaû, progetto strategico per la società Finmatica S.p.a.\newline 
  Si tratta di una piattaforma che consente lo sviluppo di applicazioni per il settore finanziario in ambiente dipartimentale. Il sistema è in grado di eseguire dei moduli software generici detti \textquotedblleft Template\textquotedblright{} che durante la fase di istanziazione vengono specializzati e personalizzati  tramite le funzionalità del framework in base alle logiche applicative meta-descritte.\newline 
  L'engine che controlla la piattaforma applicando le regole descritte dai metadati è stato progettato con metodologia UML e Object Oriented Design (OOD) e realizzato in ambiente Windows con linguaggio Visual C++, tecnologia COM per lo sviluppo dei componenti del sistema e OLE-DB per il layer di accesso ai dati.
  }
  \ecvtitle{December 1999 -- March 2001}{IT Application Analyst and Designer}
  \ecvitem{}{Treviso Tecnologia\newline{}Via Roma, 4/D, 31020 Villorba (TV)}
  \ecvitem{}{
  Progettazione e sviluppo di una presentazione multimediale sugli strumenti di eCommerce realizzata dalla C.C.I.A.A. di Treviso con tecnologia Macromedia Director.\newline
  Amministrazione e manutenzione dell'infrastruttura tecnologica basata su piattaforma Sun Solaris e Microsoft Windows NT Server (Mail Server Sendmail, DNS Bind, Web Server Apache e IIS).\newline
  Docenza per il corso di \textquotedblleft Protocolli di rete  e Networking\textquotedblright{} per gli studenti dell'istituto IPSIA di Castelfranco Veneto.\newline
  Docenza per il corso di \textquotedblleft Introduzione allo sviluppo di Applicazioni Web basate su tecnologia ASP ed accesso ai dati con ADO\textquotedblright{}.\newline
  Progettazione e sviluppo di un'applicazione per la gestione dell'anagrafica degli indirizzi per l'invio della corrispondenza e delle pubblicazioni dell'Ufficio Studi della C.C.I.A.A. di Treviso.\newline
  Analisi, progettazione e sviluppo del nuovo sistema per la gestione della Borsa Merci della C.C.I.A.A. di Treviso.
  }
  \ecvtitle{April 1996 -- November 1999}{Software Developer}
  \ecvitem{}{CadCam Studio S.r.l\newline{}Via Reginato, 87, 31100 Treviso}
  \ecvitem{}{
  Sviluppo di moduli di personalizzazione per l'utente per il software di progettazione Cad Cam realizzati in ambiente Unix HPUX con linguaggio di programmazione ANSI C.\newline
  All'interno del gruppo di lavoro del progetto ReEnge (strumento per il Reverse Engineering Meccanico) ho svolto attività di progettazione e sviluppo delle funzionalità di gestione e di editing dei modelli tridimensionali digitalizzati.\newline
  Analisi, progettazione e sviluppo di un'applicazione per la gestione del \textquotedblleft Planning delle attività aziendali\textquotedblright{} impiegando moderne tecnologie di Intranetworking e Distributed Computing (Java, JDBC, RMI).
  }
\end{en}

\begin{it}
    \ecvsection{Esperienza professionale}
    
    \ecvtitle{Maggio 2010 --  alla data attuale}{Client Manager - Project Manager - Account Manager.}
    \ecvitem{}{Gruppo Euris S.p.a.\newline Via Uruguay, 54, 35127 Padova}
    \ecvitem{}{
    Da completare \ldots
    \begin{ecvitemize}
      \item uno
      \item due
      \item tre
    \end{ecvitemize}
    }

    \ecvtitle{Luglio 2009 -- Maggio 2010}{Responsabile pianificazione delle risorse}
    \ecvitem{}{Gruppo Euris S.p.a.\newline Via Uruguay, 54, 35127 Padova}
    \ecvitem{}{
        Da completare \ldots
    }

    \ecvtitle{Luglio 2007 -- Luglio 2009}{Responsabile Tecnico per le filiali di Padova e Verona}
    \ecvitem{}{Gruppo Euris S.p.a.\newline Via Uruguay, 54, 35127 Padova}
    \ecvitem{}{
    Responsabile diretto di 45 dipendenti.\newline
    Gestione del colloquio tecnico ed inserimento delle nuove risorse nei gruppi di lavoro.\newline
    Analisi delle competenze tecniche per la fornitura dei consulenti da inserire nei progetti dei clienti.\newline
    Erogazione di corsi interni per la formazione ed il \textquotedblleft training on the job\textquotedblright{} delle risorse.\newline
    Analisi dei requisiti, analisi tecnica, stima dei tempi e dei costi, definizione di architetture software, gestione stato avanzamento lavori e change request, redazione dei vari documenti tecnici di progetto.\newline
    \ecvhighlight{Principali progetti gestiti}
    \begin{ecvitemize}
        \item Ott 2007, Redazione del documento tecnico per la partecipazione alla gara per la fornitura di un sistema software per la gestione del \textquotedblleft Sistema di Legalità\textquotedblright{} per la società InfoCamere di Padova.
        \item Feb - Mag 2008, Progettazione e Team Leader nella realizzazione di un modulo software per la gestione del \textquotedblleft Controllo di Qualità\textquotedblright{} nei processi produttivi del settore siderurgico sviluppato per un'azienda di automazioni industriali del Gruppo Marcegaglia.
        \item Mar -- Ago 2008, Progettazione e coordinamento di 4 sviluppatori nelle attività di sviluppo di un modulo di \textquotedblleft System Integration\textquotedblright{} per il sistema CRM implementato presso la società CRIF di Bologna, operante del settore Finance.
        \item Set -- Ott 2008, Progettazione e sviluppo di una Web Application per la compilazione della modulistica telematica per la gestione di rifiuti pericolosi e l'emissione di sostanze inquinanti per conto di un'azienda operante nel settore \textquotedblleft Ecologia e Ambiente\textquotedblright{} (area Pubblica Amministrazione).
        \item Nov 2008 -- Mar 2009, Consolidamento ed estensione delle funzionalità del modulo di integrazione del CRM di CRIF permettendo la gestione di informazioni provenienti da funzioni aziendali non previste nella prima fase del progetto ed estensione delle piattaforme applicative  coinvolte nel processo.
        \item Apr -- Lug 2009, Progettazione e coordinamento delle attività di sviluppo del sistema di gestione degli \textquotedblleft Incidents\textquotedblright{} nel progetto di CRM per CRIF. Realizzazione di un modulo per la gestione ed il monitoraggio del \textquotedblleft Service Level Agreement\textquotedblright.
    \end{ecvitemize}
    }
    
    \ecvtitle{Gennaio 2005 -- Luglio 2007}{Consulente Tecnico Senior per i Sistemi Informativi Direzionali}
    \ecvitem{}{Engineering S.p.a.\newline Via G. del Pian dei Carpini, 1, 50100 Firenze}
    \ecvitem{}{
    Consolidamento delle procedure di alimentazione dell'Enterprise Data Warehouse in ambiente
    dipartimentale presso il Consorzio Operativo del Gruppo Monte dei Paschi di Siena (COGMPS).\newline
    Coordinamento del gruppo di lavoro impegnato nello sviluppo degli strumenti di analisi e fruizione dei dati del EDW presso la sede del COGMPS:
    \begin{ecvitemize}
        \item Giu 2006 -- Giu 2007, Analisi, progettazione e sviluppo del Sistema di Indagine OLAP integrato. L'applicazione è stata sviluppata con metodologia agile e sviluppo iterativo utilizzando la modellazione UML per la descrizione degli \textsl{Use Cases}, degli \textsl{Activity Diagram}, dei \textsl{Sequence Diagram} e dei \textsl{Class Diagram}. Durante la progettazione è stato fatto uso dei principali design pattern, in particolare è stato applicato il CAB (Composite Application Block) definito da Pattern\&Practice di Microsoft. L'applicazione è stata realizzata in ambiente .Net, con interfaccia grafica Window Forms e business layer realizzato con tecnologia COM+.
        \item Set 2005 -- Giu 2006, Analisi e progettazione del gestore del modello dati della Banca. Applicazione di tipo Smart Client, che consente di descrivere la mappatura degli attributi delle entità che rappresentano il modello dati concettuale della banca verso le diverse rappresentazioni del modello dati relazionale implementato nelle applicazioni di ogni area.
        \item Gen 2005 -- Set 2005, Progettazione e sviluppo di un modulo software per l'esecuzione di indagini OLAP che richiedono l'uso di dati calcolati dal motore di calcolo del sistema di Batch Processing. Il sistema realizzato con tecnologia ASP.NET è in grado di inviare una richiesta ad modulo COM+ che permette di istanziare un processo batch \textquotedblleft{}on demand\textquotedblright{} e restituire i dati calcolati all'applicazione web.
        \end{ecvitemize}
    }

    \ecvtitle{Luglio 2003 -- Dicembre 2004}{Responsabile sviluppo piattaforma software area Business Intelligence}
    \ecvitem{}{Trend S.p.A.\newline Via Sorbanella, 30, 25125 Brescia}
    \ecvitem{}{
    Progettazione e coordinamento delle attività di sviluppo di applicazioni per la realizzazione di data warehouse.\newline
    Progettazione di applicazioni di Business Intelligence per il settore bancario e finanziario.
    }

    \ecvtitle{Marzo 2001 -- Giugno 2003}{Analista Programmatore Senior}
    \ecvitem{}{EURIS S.r.l.\newline Via Uruguay, 53, 35100 Padova}
    \ecvitem{}{Attività di analisi, progettazione, sviluppo e consolidamento della piattaforma di sviluppo Itaû, progetto strategico per la società Finmatica S.p.a.\newline
    Si tratta di una piattaforma che consente lo sviluppo di applicazioni per il settore finanziario in ambiente dipartimentale. Il sistema è in grado di eseguire dei moduli software generici detti \textquotedblleft Template\textquotedblright{} che durante la fase di istanziazione vengono specializzati e personalizzati  tramite le funzionalità del framework in base alle logiche applicative meta-descritte.\newline
    L'engine che controlla la piattaforma applicando le regole descritte dai metadati è stato progettato con metodologia UML e Object Oriented Design (OOD) e realizzato in ambiente Windows con linguaggio Visual C++, tecnologia COM per lo sviluppo dei componenti del sistema e OLE-DB per il layer di accesso ai dati.
    }

    \ecvtitle{Dicembre 1999 - Marzo 2001}{Progettista ed analista di sistemi informatici}
    \ecvitem{}{Treviso Tecnologia\newline{}Via Roma, 4/D, 31020 Villorba (TV)}
    \ecvitem{}{
    Progettazione e sviluppo di una presentazione multimediale sugli strumenti di eCommerce realizzata dalla C.C.I.A.A. di Treviso con tecnologia Macromedia Director.\newline
    Amministrazione e manutenzione dell'infrastruttura tecnologica basata su piattaforma Sun Solaris e Microsoft Windows NT Server (Mail Server Sendmail, DNS Bind, Web Server Apache e IIS).\newline
    Docenza per il corso di \textquotedblleft Protocolli di rete  e Networking\textquotedblright{} per gli studenti dell'istituto IPSIA di Castelfranco Veneto.\newline
    Docenza per il corso di \textquotedblleft Introduzione allo sviluppo di Applicazioni Web basate su tecnologia ASP ed accesso ai dati con ADO\textquotedblright{}.\newline
    Progettazione e sviluppo di un'applicazione per la gestione dell'anagrafica degli indirizzi per l'invio della corrispondenza e delle pubblicazioni dell'Ufficio Studi della C.C.I.A.A. di Treviso.\newline
    Analisi, progettazione e sviluppo del nuovo sistema per la gestione della Borsa Merci della C.C.I.A.A. di Treviso.
    }

    \ecvtitle{Aprile 1996 -- Dicembre 1999}{Programmatore}
    \ecvitem{}{CadCam Studio S.r.l\newline{}Via Reginato, 87, 31100 Treviso}
    \ecvitem{}{
    Sviluppo di moduli di personalizzazione per l'utente per il software di progettazione Cad Cam realizzati in ambiente Unix HPUX con linguaggio di programmazione ANSI C.\newline
    All'interno del gruppo di lavoro del progetto ReEnge (strumento per il Reverse Engineering Meccanico) ho svolto attività di progettazione e sviluppo delle funzionalità di gestione e di editing dei modelli tridimensionali digitalizzati.\newline
    Analisi, progettazione e sviluppo di un'applicazione per la gestione del \textquotedblleft Planning delle attività aziendali\textquotedblright{} impiegando moderne tecnologie di Intranetworking e Distributed Computing (Java, JDBC, RMI).
    }
\end{it}
  
%----------------------------------------------------------------------------------------
% EDUCATION AND TRAINING FOR SECTION
%----------------------------------------------------------------------------------------
\begin{en}
\ecvsection[10pt]{Education and training}
\ecvtitle{1995 -- 1999}{Laurea Triennale in Ingegneria Informatica}
\ecvitem{}{Università degli Studi di Padova\newline{}Via 8 Febbraio, 2, 35122 Padova
}

\ecvtitle{1986 -- 1991}{Secondary school certificate}
\ecvitem{}{
Istituto Tecnico Industriale C. Zuccante\newline{}Via Baglioni, 22, 30173 Mestre (VE)
}
\ecvitem{}{
\begin{ecvitemize}
\item Elettronica 
\item Elettronica digitale
\item Informatica
\item Telecomunicazioni
\end{ecvitemize}
}
\end{en}

\begin{it}
\ecvsection[10pt]{Istruzione e formazione}
\ecvtitle{1995 -- 1999}{Laurea Triennale in Ingegneria Informatica}
\ecvitem{}{Università degli Studi di Padova\newline{}Via 8 Febbraio, 2, 35122 Padova
}

\ecvtitle{1986 -- 1991}{Diploma di Perito Industriale Capotecnico. Specialità Elettronica Industriale}
\ecvitem{}{
Istituto Tecnico Industriale C. Zuccante\newline{}Via Baglioni, 22, 30173 Mestre (VE)
}
\ecvitem{}{
\begin{ecvitemize}
\item Elettronica 
\item Elettronica digitale
\item Informatica
\item Telecomunicazioni
\end{ecvitemize}
}
\end{it}

%----------------------------------------------------------------------------------------
% PERSONAL SKILLS SECTION
%----------------------------------------------------------------------------------------
\begin{en}
\ecvsection{Personal skills}
\ecvmothertongue{Italian}
%\ecvitem{\large Altra/e lingua/e}{}
\ecvlanguageheader{}
\ecvlanguage{Inglese}{B2}{C1}{B1}{B1}{B2}
\ecvlastlanguage{Francese}{A1}{A2}{A1}{A1}{A1}
\ecvlanguagefooter{}

\ecvblueitem{Communication skills}{
Buona predisposizione ai rapporti interpersonali sviluppata sia in attività lavorative (come analista programmatore o come Team Leader) sia in attività sportive (basket, escursionismo, arrampicata, motociclismo).\newline
Buone capacità di comunicazione e di negoziazione sia all'interno del gruppo di lavoro, che nei confronti dei referenti del cliente, maturate durante lunghi periodi di permanenza presso le sedi dei clienti stessi.
}
\ecvblueitem{Organisational / managerial skills}{
Capacità di leadership conseguita nella gestione dei 45 dipendenti delle filiali di Padova e Verona.
Maturata buona esperienza nell'organizzazione di gruppi di lavoro e di progetti di medie dimensioni.
}

  \ecvdigitalcompetence
      {\ecvProficient}
      {\ecvProficient}
      {\ecvProficient}
      {\ecvProficient}
      {\ecvProficient}

\ecvblueitem{Computer skills}{
\textbf{Sistemi Operativi}\newline
\textbullet\hspace{5pt}Buona conoscenza dei sistemi operativi Windows 98/NT/2000/XP/2003 Server e delle loro interazioni con il linguaggio C/C++ e con il Framework di sviluppo .NET, buone anche le competenze sistemistiche.\newline
\textbullet\hspace{5pt}Utilizzatore avanzato con buone competenze sistemistiche del S.O. Linux.\newline
{}\newline
\textbf{Basi di Dati}\newline
\textbullet\hspace{5pt}Buona conoscenza del linguaggio SQL.\newline
\textbullet\hspace{5pt}Consolidata esperienza nella programmazione di applicazioni basate su MS SQL Server.\newline
\textbullet\hspace{5pt}Conoscenze di base dei DBMS Oracle e del linguaggio PL/SQL, MySQL e Postgres.\newline
{}\newline
\textbf{Linguaggi di programmazione}\newline
\textbullet\hspace{5pt}Buona conoscenza del Framework .NET e del linguaggio C\#.\newline
\textbullet\hspace{5pt}Conoscenza avanzata del linguaggio C/C++ e delle principali API per lo sviluppo di applicazioni grafiche (MFC, GDI, OpenGL).\newline
\textbullet\hspace{5pt}Consolidata esperienza nello sviluppo di architetture per l'accesso ai dati tramite le ODBC, ADO e del modello provider-consumer di OleDB.\newline
\textbullet\hspace{5pt}Buona conoscenza di strumenti per lo sviluppo di Web Application (HTML, DHTML, XML, ASP, PHP, JavaScript e VBScript).\newline
{}\newline
\textbf{Metodologie di progettazione OO}\newline
\textbullet\hspace{5pt}Buona esperienza nella progettazione di applicazioni desktop e di web application utilizzando architetture N--Tier e SOA.\newline
\textbullet\hspace{5pt}Esperienza nell'applicazione di Design Pattern e nell'uso della modellazione UML.\newline
\textbullet\hspace{5pt}Conoscenze delle metodologia Agile, Domain Driven Design e delle metodologia UP e RUP per la gestione dei progetti.\newline
}

\ecvblueitem{Driving licence}{A,B}
\end{en}

\begin{it}
\ecvsection{Competenze Personali}
\ecvmothertongue{Italiano}
%\ecvitem{\large Altra/e lingua/e}{}
\ecvlanguageheader{}
\ecvlanguage{Inglese}{B2}{C1}{B1}{B1}{B2}
\ecvlastlanguage{Francese}{A1}{A2}{A1}{A1}{A1}
\ecvlanguagefooter{}

\ecvblueitem{Competenze comunicative}{
Buona predisposizione ai rapporti interpersonali sviluppata sia in attività lavorative (come analista programmatore o come Team Leader) sia in attività sportive (basket, escursionismo, arrampicata, motociclismo).\newline
Buone capacità di comunicazione e di negoziazione sia all'interno del gruppo di lavoro, che nei confronti dei referenti del cliente, maturate durante lunghi periodi di permanenza presso le sedi dei clienti stessi.
}
\ecvblueitem{Competenze organizzative e gestionali}{
Capacità di leadership conseguita nella gestione dei 45 dipendenti delle filiali di Padova e Verona.
Maturata buona esperienza nell'organizzazione di gruppi di lavoro e di progetti di medie dimensioni.
}

%----------------------------------
% COMPETENZE DIGITALI
% In riferimento a:
% europass.cedefop.europa.eu/it/resources/digital-competences
% I livelli sono così definiti:
%       Utente Base			--> \ecvBasic
%       Utente Autonomo		--> \ecvIndependent
%       Utente Avanzato		--> \ecvProficient
%
% E suddivisi in 5 aree di competenza:
%       Eaborazione delle informazioni
%       Creazione di Contenuti
%       Comunicazione
%       Risoluzione di problemi
%       Sicurezza
%----------------------------------
  \ecvdigitalcompetence
      {\ecvProficient}
      {\ecvProficient}
      {\ecvProficient}
      {\ecvProficient}
      {\ecvProficient}

\ecvblueitem{Competenze informatiche}{
\textbf{Sistemi Operativi}\newline
\textbullet\hspace{5pt}Buona conoscenza dei sistemi operativi Windows 98/NT/2000/XP/2003 Server e delle loro interazioni con il linguaggio C/C++ e con il Framework di sviluppo .NET, buone anche le competenze sistemistiche.\newline
\textbullet\hspace{5pt}Utilizzatore avanzato con buone competenze sistemistiche del S.O. Linux.\newline
{}\newline
\textbf{Basi di Dati}\newline
\textbullet\hspace{5pt}Buona conoscenza del linguaggio SQL.\newline
\textbullet\hspace{5pt}Consolidata esperienza nella programmazione di applicazioni basate su MS SQL Server.\newline
\textbullet\hspace{5pt}Conoscenze di base dei DBMS Oracle e del linguaggio PL/SQL, MySQL e Postgres.\newline
{}\newline
\textbf{Linguaggi di programmazione}\newline
\textbullet\hspace{5pt}Buona conoscenza del Framework .NET e del linguaggio C\#.\newline
\textbullet\hspace{5pt}Conoscenza avanzata del linguaggio C/C++ e delle principali API per lo sviluppo di applicazioni grafiche (MFC, GDI, OpenGL).\newline
\textbullet\hspace{5pt}Consolidata esperienza nello sviluppo di architetture per l'accesso ai dati tramite le ODBC, ADO e del modello provider-consumer di OleDB.\newline
\textbullet\hspace{5pt}Buona conoscenza di strumenti per lo sviluppo di Web Application (HTML, DHTML, XML, ASP, PHP, JavaScript e VBScript).\newline
{}\newline
\textbf{Metodologie di progettazione OO}\newline
\textbullet\hspace{5pt}Buona esperienza nella progettazione di applicazioni desktop e di web application utilizzando architetture N--Tier e SOA.\newline
\textbullet\hspace{5pt}Esperienza nell'applicazione di Design Pattern e nell'uso della modellazione UML.\newline
\textbullet\hspace{5pt}Conoscenze delle metodologia Agile, Domain Driven Design e delle metodologia UP e RUP per la gestione dei progetti.\newline
}

\ecvblueitem{Patente di guida}{A,B}
\end{it}

\begin{extraInfo}
%----------------------------------------------------------------------------------------
% ADDITIONAL INFORMATION FOR SECTION
% Replace with relevant publications, presentations, projects, conferences, seminars,
% honours and awards, memberships, references.
%----------------------------------------------------------------------------------------
\begin{en}
\ecvsection{Additional information}
\ecvitem[10pt]{Corsi frequentati}{
\textbullet\hspace{5pt}Gennaio 2008 -- Corso di \textquotedblleft Leadership Generativa\textquotedblright{} della durata di 3 giorni sulle basi della comunicazione, i valori e la mission personale, stile di leadership e processo di delega, erogato da Comteam di Roma.\newline
\textbullet\hspace{5pt}Aprile 2008  -- Corso di \textquotedblleft Tecniche di Comunicazione Efficace\textquotedblright{} della durata di 3 giorni sui seguenti argomenti: I principi della comunicazione, I tre livelli della comunicazione, Ascolto Attivo ed Empatia, I sistemi rappresentazionali, I segnali subliminali, Ricalco e Guida, erogato da Comteam di Roma.\newline
\textbullet\hspace{5pt}Luglio 2008  -- Corso di \textquotedblleft Gestione dei Conflitti\textquotedblright{} della durata di 3 giorni sui seguenti temi: le origini del conflitto: lo scontro delle mappe, la risoluzione del conflitto riportando a livello conscio i 3 modellatori universali: generalizzazioni, distorsioni e cancellazioni, erogato da Comteam di Roma.\newline
\textbullet\hspace{5pt}Dicembre 2008 -- Corso \textquotedblleft KnowHow Lab\textquotedblright{}, Seminario sui Metaprogrammi e la PNL (Programmazione Neuro--Linguistica) erogato da Comteam di Roma.
}
\end{en}

\begin{it}
\ecvsection{Ulteriori informazioni}
\ecvitem[10pt]{Corsi frequentati}{
\textbullet\hspace{5pt}Gennaio 2008 -- Corso di \textquotedblleft Leadership Generativa\textquotedblright{} della durata di 3 giorni sulle basi della comunicazione, i valori e la mission personale, stile di leadership e processo di delega, erogato da Comteam di Roma.\newline
\textbullet\hspace{5pt}Aprile 2008  -- Corso di \textquotedblleft Tecniche di Comunicazione Efficace\textquotedblright{} della durata di 3 giorni sui seguenti argomenti: I principi della comunicazione, I tre livelli della comunicazione, Ascolto Attivo ed Empatia, I sistemi rappresentazionali, I segnali subliminali, Ricalco e Guida, erogato da Comteam di Roma.\newline
\textbullet\hspace{5pt}Luglio 2008  -- Corso di \textquotedblleft Gestione dei Conflitti\textquotedblright{} della durata di 3 giorni sui seguenti temi: le origini del conflitto: lo scontro delle mappe, la risoluzione del conflitto riportando a livello conscio i 3 modellatori universali: generalizzazioni, distorsioni e cancellazioni, erogato da Comteam di Roma.\newline
\textbullet\hspace{5pt}Dicembre 2008 -- Corso \textquotedblleft KnowHow Lab\textquotedblright{}, Seminario sui Metaprogrammi e la PNL (Programmazione Neuro--Linguistica) erogato da Comteam di Roma.
}
\end{it}

%----------------------------------------------------------------------------------------
% ANNEXES
% Replace with list of documents annexed to your CV
% (copies of degrees and qualifications, testimonial of employment or work placement, publications or research)
%----------------------------------------------------------------------------------------

\end{extraInfo}

%----------------------------------------------------------------------------------------
% PRIVACY LAW (only for Italian version)
%----------------------------------------------------------------------------------------
\begin{it}
  \ecvblueitem[15pt]{}{Autorizzo il trattamento dei dati personali contenuti nel mio curriculum vitae in base all’art. 13 del D.Lgs. 196/2003 e all’art. 13 del Regolamento UE 2016/679 relativo alla protezione delle persone fisiche con riguardo al trattamento dei dati personali.}
  
  %\ecvsection{Legge Privacy}
  %\ecvitem{Consenso al trattamento dei dati personali}{Il sottoscritto Zennaro Andrea autorizza all'utilizzo dei dati presenti in questo curriculum per soli scopi di selezione in virtù di quanto previsto dal Decreto legislativo 30 giugno 2003 n. 196.}
\end{it}

%\bibliographystyle{plain}
%\nobibliography{publications}
%\ecvitem{}{\textbf{Pubblicazioni}}
%\ecvitem{}{\bibentry{pub1}}
%\ecvitem[10pt]{}{\bibentry{pub2}}
%\ecvitem{}{\textbf{Interessi personali}}
%\ecvitem{}{\ldots}

%\ecvsection{Allegati}
%\ecvitem{}{Enumerare gli allegati al CV.}

\end{europasscv}
\end{document}
