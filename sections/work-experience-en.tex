

  \ecvsection{Work Experience}
  \ecvtitle{May 2010 --  }{Client Manager - Project Manager - Account Manager }
  \ecvitem{}{Gruppo Euris S.p.a.\newline Via Uruguay, 54, 35127 Padova}
  \ecvitem{}{
    \begin{ecvitemize}
        \item in a primary Italian Insurance Group, lead an Application Management service managing a team based up to 60 resources being responsible of SLAs and KPIs monitoring and reporting, Project Planning and accounting, budget monitoring.
        \item in a primary Italian Insurance Group lead the development of AI Predictive Services platform based on target Cloud Environment (Openshift, Docker, Kubernetes).
        \item in a primary Italian Insurance Group, project management and process analysis for the implementation of Motor Insurance Telematics products
        \item in primary Italian Insurance Group, project management and process analysis for the implementation of portfolio integration and policy acquisition with Financial Services partners of primary automotive group. 
        \item in primary Italian Insurance Group, project management for the new agency Front End web application Digital Agency.
        \item in a primary Italian Insurance Group, business process analysis and team leading to implement a web interface application for insurance premium collection
        \item in a primary Italian Credit Risk Services company, lead a CRM integration project, merging two legacy CRM platform and building a new enterprise CRM solution.
        \item in primary Italian environment information system agency, development of a Web Application using Model Driven Design approach to collect Environment Pollution data from Industrial activities. 
        \item in a leading industrial group in the steel processing sector, development of a Web application to monitor real time control process in Plant Automation sector.
    \end{ecvitemize}
   }
  \ecvtitle{July 2009 -- May 2010}{Responsabile pianificazione delle risorse}
  \ecvitem{}{Gruppo Euris S.p.a.\newline Via Uruguay, 54, 35127 Padova}
  \ecvitem{}{
  Da completare \ldots
  }

  \ecvtitle{July 2007 -- July 2009}{Responsabile Tecnico per le filiali di Padova e Verona}
  \ecvitem{}{Gruppo Euris S.p.a.\newline Via Uruguay, 54, 35127 Padova}
  \ecvitem{}{
  Responsabile diretto di 45 dipendenti.\newline
  Gestione del colloquio tecnico ed inserimento delle nuove risorse nei gruppi di lavoro.\newline
  Analisi delle competenze tecniche per la fornitura dei consulenti da inserire nei progetti dei clienti.\newline
  Erogazione di corsi interni per la formazione ed il \textquotedblleft training on the job\textquotedblright{} delle risorse.\newline
  Analisi dei requisiti, analisi tecnica, stima dei tempi e dei costi, definizione di architetture software, gestione stato avanzamento lavori e change request, redazione dei vari documenti tecnici di progetto.\newline
  \ecvhighlight{Principali progetti gestiti}
  \begin{ecvitemize}
  \item Ott 2007, Redazione del documento tecnico per la partecipazione alla gara per la fornitura di un sistema software per la gestione del \textquotedblleft Sistema di Legalità\textquotedblright{} per la società InfoCamere di Padova.
  \item Feb - Mag 2008, Progettazione e Team Leader nella realizzazione di un modulo software per la gestione del \textquotedblleft Controllo di Qualità\textquotedblright{} nei processi produttivi del settore siderurgico sviluppato per un'azienda di automazioni industriali del Gruppo Marcegaglia.
  \item Mar -- Ago 2008, Progettazione e coordinamento di 4 sviluppatori nelle attività di sviluppo di un modulo di \textquotedblleft System Integration\textquotedblright{} per il sistema CRM implementato presso la società CRIF di Bologna, operante del settore Finance.
  \item Set -- Ott 2008, Progettazione e sviluppo di una Web Application per la compilazione della modulistica telematica per la gestione di rifiuti pericolosi e l'emissione di sostanze inquinanti per conto di un'azienda operante nel settore \textquotedblleft Ecologia e Ambiente\textquotedblright{} (area Pubblica Amministrazione).
  \item Nov 2008 -- Mar 2009, Consolidamento ed estensione delle funzionalità del modulo di integrazione del CRM di CRIF permettendo la gestione di informazioni provenienti da funzioni aziendali non previste nella prima fase del progetto ed estensione delle piattaforme applicative  coinvolte nel processo.
  \item Apr -- Lug 2009, Progettazione e coordinamento delle attività di sviluppo del sistema di gestione degli \textquotedblleft Incidents\textquotedblright{} nel progetto di CRM per CRIF. Realizzazione di un modulo per la gestione ed il monitoraggio del \textquotedblleft Service Level Agreement\textquotedblright.
  \end{ecvitemize}
  }
  \ecvtitle{January 2005 -- July 2007}{Senior Consultant}
  \ecvitem{}{Engineering S.p.a.\newline Via G. del Pian dei Carpini, 1, 50100 Firenze}
  \ecvitem{}{
  Consolidamento delle procedure di alimentazione dell'Enterprise Data Warehouse in ambiente
  dipartimentale presso il Consorzio Operativo del Gruppo Monte dei Paschi di Siena (COGMPS).\newline
  Coordinamento del gruppo di lavoro impegnato nello sviluppo degli strumenti di analisi e fruizione dei dati del EDW presso la sede del COGMPS:
  \begin{ecvitemize}
  \item Giu 2006 -- Giu 2007, Analisi, progettazione e sviluppo del Sistema di Indagine OLAP integrato. L'applicazione è stata sviluppata con metodologia agile e sviluppo iterativo utilizzando la modellazione UML per la descrizione degli \textsl{Use Cases}, degli \textsl{Activity Diagram}, dei \textsl{Sequence Diagram} e dei \textsl{Class Diagram}. Durante la progettazione è stato fatto uso dei principali design pattern, in particolare è stato applicato il CAB (Composite Application Block) definito da Pattern\&Practice di Microsoft. L'applicazione è stata realizzata in ambiente .Net, con interfaccia grafica Window Forms e business layer realizzato con tecnologia COM+.
  \item Set 2005 -- Giu 2006, Analisi e progettazione del gestore del modello dati della Banca. Applicazione di tipo Smart Client, che consente di descrivere la mappatura degli attributi delle entità che rappresentano il modello dati concettuale della banca verso le diverse rappresentazioni del modello dati relazionale implementato nelle applicazioni di ogni area.
  \item Gen 2005 -- Set 2005, Progettazione e sviluppo di un modulo software per l'esecuzione di indagini OLAP che richiedono l'uso di dati calcolati dal motore di calcolo del sistema di Batch Processing. Il sistema realizzato con tecnologia ASP.NET è in grado di inviare una richiesta ad modulo COM+ che permette di istanziare un processo batch \textquotedblleft{}on demand\textquotedblright{} e restituire i dati calcolati all'applicazione web.
  \end{ecvitemize}
  }
  \ecvtitle{July 2003 -- December 2004}{Technical Leader for Business Intelligence Application}
  \ecvitem{}{Trend S.p.A.\newline Via Sorbanella, 30, 25125 Brescia}
  \ecvitem{}{
  Progettazione e coordinamento delle attività di sviluppo di applicazioni per la realizzazione di data warehouse.\newline
  Progettazione di applicazioni di Business Intelligence per il settore bancario e finanziario.
  }
  \ecvtitle{March 2001 -- June 2003}{Senior Software Engineer}
  \ecvitem{}{EURIS S.r.l.\newline Via Uruguay, 53, 35100 Padova}
  \ecvitem{}{
  Attività di analisi, progettazione, sviluppo e consolidamento della piattaforma di sviluppo Itaû, progetto strategico per la società Finmatica S.p.a.\newline 
  Si tratta di una piattaforma che consente lo sviluppo di applicazioni per il settore finanziario in ambiente dipartimentale. Il sistema è in grado di eseguire dei moduli software generici detti \textquotedblleft Template\textquotedblright{} che durante la fase di istanziazione vengono specializzati e personalizzati  tramite le funzionalità del framework in base alle logiche applicative meta-descritte.\newline 
  L'engine che controlla la piattaforma applicando le regole descritte dai metadati è stato progettato con metodologia UML e Object Oriented Design (OOD) e realizzato in ambiente Windows con linguaggio Visual C++, tecnologia COM per lo sviluppo dei componenti del sistema e OLE-DB per il layer di accesso ai dati.
  }
  \ecvtitle{December 1999 -- March 2001}{IT Application Analyst and Designer}
  \ecvitem{}{Treviso Tecnologia\newline{}Via Roma, 4/D, 31020 Villorba (TV)}
  \ecvitem{}{
  Progettazione e sviluppo di una presentazione multimediale sugli strumenti di eCommerce realizzata dalla C.C.I.A.A. di Treviso con tecnologia Macromedia Director.\newline
  Amministrazione e manutenzione dell'infrastruttura tecnologica basata su piattaforma Sun Solaris e Microsoft Windows NT Server (Mail Server Sendmail, DNS Bind, Web Server Apache e IIS).\newline
  Docenza per il corso di \textquotedblleft Protocolli di rete  e Networking\textquotedblright{} per gli studenti dell'istituto IPSIA di Castelfranco Veneto.\newline
  Docenza per il corso di \textquotedblleft Introduzione allo sviluppo di Applicazioni Web basate su tecnologia ASP ed accesso ai dati con ADO\textquotedblright{}.\newline
  Progettazione e sviluppo di un'applicazione per la gestione dell'anagrafica degli indirizzi per l'invio della corrispondenza e delle pubblicazioni dell'Ufficio Studi della C.C.I.A.A. di Treviso.\newline
  Analisi, progettazione e sviluppo del nuovo sistema per la gestione della Borsa Merci della C.C.I.A.A. di Treviso.
  }
  \ecvtitle{April 1996 -- November 1999}{Software Developer}
  \ecvitem{}{CadCam Studio S.r.l\newline{}Via Reginato, 87, 31100 Treviso}
  \ecvitem{}{
  Sviluppo di moduli di personalizzazione per l'utente per il software di progettazione Cad Cam realizzati in ambiente Unix HPUX con linguaggio di programmazione ANSI C.\newline
  All'interno del gruppo di lavoro del progetto ReEnge (strumento per il Reverse Engineering Meccanico) ho svolto attività di progettazione e sviluppo delle funzionalità di gestione e di editing dei modelli tridimensionali digitalizzati.\newline
  Analisi, progettazione e sviluppo di un'applicazione per la gestione del \textquotedblleft Planning delle attività aziendali\textquotedblright{} impiegando moderne tecnologie di Intranetworking e Distributed Computing (Java, JDBC, RMI).
  }