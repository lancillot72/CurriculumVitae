
    \ecvsection{Esperienza professionale}
    
    \ecvtitle{Aprile 2018 -- alla data attuale}{IT Client Manager - Service Manager.}
    \ecvitem{}{Gruppo Euris S.p.a.\newline Via Uruguay, 54, 35127 Padova}
    \ecvitem{}{
      IT Client Manager presso un importante cliente del settore assicurativo con responsabilità nella gestione dei livelli di servizio e delle performance dei gruppi di lavoro impiegati in un servizio di Application Maintenance per le aree Auto e Danni sviluppando le seguenti competenze:
    \begin{ecvitemize}
      \item Definizione dei KPI di servizio e monitoraggio degli SLA contrattuali. 
      \item Pianificazione dei carichi di lavoro e allocazione delle risorse.
      \item Produzione della documentazione per le service review periodiche.
      \item Gestione delle aspettative e delle relazioni con il cliente. 
    \end{ecvitemize}
    }

    \ecvtitle{Maggio 2010 -- Aprile 2018}{IT Project Manager.}
    \ecvitem{}{Gruppo Euris S.p.a.\newline Via Uruguay, 54, 35127 Padova}
    \ecvitem{}{
      IT Project Manager con comprovata esperienza nella gestione dell'intero ciclo di vita dei progetti presso un importante cliente del settore assicurativo dimostrando capacità di leadership e team building sia nelle gestione contemporanea di più progetti di piccole dimensioni che nella gestione di progetti di grandi dimensioni.
    \begin{ecvitemize}
      \item Tutti i progetti sono stati chiusi nel rispetto dei tempi e del costi previsti e con la massima soddisfazione da parte degli stake holder . 
      \item Esperienza nel guidare ed indirizzare le decisioni di business in contesti cross funzionali su complessi progetti di integrazione.
      \item Ottime capacità di coordinamento maturate in progetti internazionali con importanti brand del settore Automotive (BMW, Ford, General Motors).
    \end{ecvitemize}
    }

    \ecvtitle{Luglio 2007 -- Maggio 2010}{Responsabile Tecnico per le filiali di Padova e Verona}
    \ecvitem{}{Gruppo Euris S.p.a.\newline Via Uruguay, 54, 35127 Padova}
    \ecvitem{}{
    
    Nel ruolo di Responsabile Tecnico, a riporto del direttore della Business Unit, con responsabilità nella selezione delle risorse ed organizzazione dei gruppi di lavoro da inserire nei progetti dei clienti.
    Formazione e \textquotedblleft training on the job\textquotedblright{} delle risorse.\newline
    Analisi dei requisiti, analisi tecnica, stima dei tempi e dei costi, definizione di architetture software, gestione stato avanzamento lavori e change request, redazione dei vari documenti tecnici di progetto.\newline
        %\ecvhighlight{Principali progetti gestiti}
        %\begin{ecvitemize}
        %    \item Redazione del documento tecnico per la partecipazione alla gara per la fornitura di un sistema software per la gestione del \textquotedblleft Sistema di Legalità\textquotedblright{} per la società InfoCamere di Padova.
        %    \item Progettazione e Team Leading per la realizzazione di un modulo software per la gestione del \textquotedblleft Controllo di Qualità\textquotedblright{} nei processi produttivi del settore siderurgico sviluppato per un'azienda di automazioni industriali del Gruppo Marcegaglia.
        %    \item Progettazione e coordinamento di 4 sviluppatori nelle attività di sviluppo di un modulo di \textquotedblleft System Integration\textquotedblright{} per il sistema CRM implementato presso la società CRIF di Bologna, operante del settore Finance.
        %    \item Progettazione e sviluppo di una Web Application per la compilazione della modulistica telematica per la gestione di rifiuti pericolosi e l'emissione di sostanze inquinanti per conto di un'azienda operante nel settore \textquotedblleft Ecologia e Ambiente\textquotedblright{} (area Pubblica Amministrazione).
        %    \item Consolidamento ed estensione delle funzionalità del modulo di integrazione del CRM di CRIF permettendo la gestione di informazioni provenienti da funzioni aziendali non previste nella prima fase del progetto ed estensione delle piattaforme applicative coinvolte nel processo.
        %    \item Progettazione e coordinamento delle attività di sviluppo del sistema di gestione degli \textquotedblleft Incidents\textquotedblright{} nel progetto di CRM per CRIF. Realizzazione di un modulo per la gestione ed il monitoraggio del \textquotedblleft Service Level Agreement\textquotedblright.
        %\end{ecvitemize}
    }

    
    \ecvtitle{Gennaio 2005 -- Luglio 2007}{Consulente Senior per i Sistemi Informativi Direzionali}
    \ecvitem{}{Engineering S.p.a.\newline Via G. del Pian dei Carpini, 1, 50100 Firenze}
    \ecvitem{}{
    Nel ruolo di consulente senior impiegato nelle attività di consolidamento delle procedure di alimentazione dell'Enterprise Data Warehouse in ambiente
    dipartimentale presso il Consorzio Operativo del Gruppo Monte dei Paschi di Siena (COGMPS).\newline
    Coordinamento del gruppo di lavoro impegnato nello sviluppo degli strumenti di analisi e fruizione dei dati del EDW presso la sede del COGMPS.
    
    %\begin{ecvitemize}
    %    \item Analisi, progettazione e sviluppo del Sistema di Indagine OLAP integrato. L'applicazione è stata sviluppata con metodologia agile e sviluppo iterativo utilizzando la modellazione UML per la descrizione degli \textsl{Use Cases}, degli \textsl{Activity Diagram}, dei \textsl{Sequence Diagram} e dei \textsl{Class Diagram}. Durante la progettazione è stato fatto uso dei principali design pattern, in particolare è stato applicato il CAB (Composite Application Block) definito da Pattern\&Practice di Microsoft. L'applicazione è stata realizzata in ambiente .Net, con interfaccia grafica Window Forms e business layer realizzato con tecnologia COM+.
    %   \item Analisi e progettazione del gestore del modello dati della Banca. Applicazione di tipo Smart Client, che consente di descrivere la mappatura degli attributi delle entità che rappresentano il modello dati concettuale della banca verso le diverse rappresentazioni del modello dati relazionale implementato nelle applicazioni di ogni area.
    %    \item Progettazione e sviluppo di un modulo software per l'esecuzione di indagini OLAP che richiedono l'uso di dati calcolati dal motore di calcolo del sistema di Batch Processing. Il sistema realizzato con tecnologia ASP.NET è in grado di inviare una richiesta ad modulo COM+ che permette di istanziare un processo batch \textquotedblleft{}on demand\textquotedblright{} e restituire i dati calcolati all'applicazione web.
    %\end{ecvitemize}
    }

    \ecvtitle{Luglio 2003 -- Dicembre 2004}{Responsabile sviluppo piattaforma software area Business Intelligence}
    \ecvitem{}{Trend S.p.A.\newline Via Sorbanella, 30, 25125 Brescia}
    \ecvitem{}{
    Progettazione e coordinamento delle attività di sviluppo di applicazioni per la realizzazione di data warehouse.\newline
    Progettazione di applicazioni di Business Intelligence per il settore bancario e finanziario.
    }

    \ecvtitle{Marzo 2001 -- Giugno 2003}{Analista Programmatore Senior}
    \ecvitem{}{EURIS S.r.l.\newline Via Uruguay, 53, 35100 Padova}
    \ecvitem{}{Attività di analisi, progettazione, sviluppo e consolidamento della piattaforma di sviluppo Itaû, progetto strategico per la società Finmatica S.p.a.\newline
    Si tratta di una piattaforma che consente lo sviluppo di applicazioni per il settore finanziario in ambiente dipartimentale. Il sistema è in grado di eseguire dei moduli software generici detti \textquotedblleft Template\textquotedblright{} che durante la fase di istanziazione vengono specializzati e personalizzati tramite le funzionalità del framework in base alle logiche applicative meta-descritte.\newline
    L'engine che controlla la piattaforma applicando le regole descritte dai metadati è stato progettato con metodologia UML e Object Oriented Design (OOD) e realizzato in ambiente Windows con linguaggio Visual C++, tecnologia COM per lo sviluppo dei componenti del sistema e OLE-DB per il layer di accesso ai dati.
    }

    \ecvtitle{Dicembre 1999 - Marzo 2001}{Progettista ed analista di sistemi informatici}
    \ecvitem{}{Treviso Tecnologia\newline{}Via Roma, 4/D, 31020 Villorba (TV)}
    \ecvitem{}{
    Progettazione e sviluppo di una presentazione multimediale sugli strumenti di eCommerce realizzata dalla C.C.I.A.A. di Treviso con tecnologia Macromedia Director.\newline
    Amministrazione e manutenzione dell'infrastruttura tecnologica basata su piattaforma Sun Solaris e Microsoft Windows NT Server (Mail Server Sendmail, DNS Bind, Web Server Apache e IIS).\newline
    Docenza per il corso di \textquotedblleft Protocolli di rete e Networking\textquotedblright{} per gli studenti dell'istituto IPSIA di Castelfranco Veneto.\newline
    Docenza per il corso di \textquotedblleft Introduzione allo sviluppo di Applicazioni Web basate su tecnologia ASP ed accesso ai dati con ADO\textquotedblright{}.\newline
    Progettazione e sviluppo di un'applicazione per la gestione dell'anagrafica degli indirizzi per l'invio della corrispondenza e delle pubblicazioni dell'Ufficio Studi della C.C.I.A.A. di Treviso.\newline
    Analisi, progettazione e sviluppo del nuovo sistema informativo per la gestione della Borsa Merci della C.C.I.A.A. di Treviso.
    }
    
    \ecvtitle{Aprile 1996 -- Dicembre 1999}{Programmatore}
    \ecvitem{}{CadCam Studio S.r.l\newline{}Via Reginato, 87, 31100 Treviso}
    \ecvitem{}{
    Sviluppo di moduli di personalizzazione per l'utente per il software di progettazione Cad Cam realizzati in ambiente Unix HPUX con linguaggio di programmazione ANSI C.
    \begin{ecvitemize}
        \item Acquisito competenze base di team working e degli strumenti di versionamento e gestione dei sorgenti.
        \item Maturata esperienza nella stima dell'implementazione del software.
        \item Consolidata la conoscenza del OOP e delle metodologie di profiliazione del codice (Perfomance Profiling).
    \end{ecvitemize}
           
    % All'interno del gruppo di lavoro del progetto ReEnge (strumento per il Reverse Engineering Meccanico) ho svolto attività di progettazione e sviluppo delle funzionalità di gestione e di editing dei modelli tridimensionali digitalizzati.\newline
    % Analisi, progettazione e sviluppo di un'applicazione per la gestione del \textquotedblleft Planning delle attività aziendali\textquotedblright{} impiegando moderne tecnologie di Intranetworking e Distributed Computing (Java, JDBC, RMI).
    }
